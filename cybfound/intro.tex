\chapter*{Введение}
На рубеже XIX--XX веков в среде математиков все популярнее начали становиться разговоры о кризисе оснований их родной науки. Несмотря на то, что удовлетворительные ответы на поставленные в те времена вопросы так и не были получены, слово <<кризис>> перестало использоваться в контексте обсуждения основ математической теории. Этому есть оправдание: кризис, который продолжается почти сотню лет, становится нормой. Норма при этом вовсе не обязана быть действительно хорошим состоянием дел. <<Кризис>> до сих пор существует. Яркой иллюстрацией такого положения является замечательный художественный образ <<идеального математика>> наших дней, представленный в книге <<Математический опыт>> Ф. Дж. Дэвиса и Р. Херша. Сами будучи довольно известными математиками, авторы используют слово <<идеальный>> вовсе не потому, что их герой является хоть в каком-то смысле совершенным. <<Идеальный математик>> является идеальным представителем всех остальных математиков, так похожих на него. Он <<наиболее математикообразный математик>>. Его воображаемая область научных интересов это <<неримановы гиперквадраты>>, он всецело предан предмету своих исследований. Он сутками изучает их, а его жизнь успешна и наполнена смыслом ровно до того момента, пока он не перестанет получать новые знания об этих объектах.

Кризис оснований математики имеет два различных аспекта, отображенных в образе <<идеального математика>>: первый имеет отношение к природе и самому существованию изучаемых им сущностей, а второй ставит вопрос о причине, по которой эти сущности вообще необходимо изучать. Процитируем первоисточник:
\begin{quotation}
  Объекты, которые изучает наш математик, были неизвестны до начала двадцатого столетия; скорее всего о них никто не знал даже тридцать лет назад, однако сегодня они занимают центральное место в списке интересов нескольких десятков (максимум, нескольких сотен) его коллег. Математик и его единомышленники даже не сомневаются в том, что неримановы гиперквадраты являются частью актуальной действительности как, например, Гибралтар или комета Галлея, более того, есть доказательство существования неримановых гиперквадратов --- важнейшее достижение этой группы людей --- в то время как существование Гибралтара вполне возможно, но не строго не доказано.

  ... Вера нашего математика опирается на метод строгого доказательства; он уверен, что существует разница между правильным и неправильным доказательством, причем эта разница весьма существенна и имеет решающее значение... Правда, он пока не может дать достаточно точное определени строгости, также как и не может сказать, что делает доказательство строгим. В его собственных работах граница между полным и неполным доказательствов довольно расплывчата и часто противоречива.
\end{quotation}

О методах математической науки авторы рассуждают следующим образом:
\begin{quotation}
  Математики твердо уверены, что они изучают объективную реальность. Для стороннего человека, однако, они могут выглядеть вовлеченными в некую эзотерическую связь с самими собой и небольшой группой друзей. Как мы, математики, можем убедить скептически настроенного незнакомца в том, что наши теоремы имеют хоть какое-то значение в мире, существующем за пределами стен нашего братства?

  Если этот человек примет наш <<устав>> и наши методы и проведет два или три года, изучая математику на уровне хотя бы аспирантуры, он вполне сможет принять наш способ мышления и перестать быть скептически настроенным незнакомцем. Точно так же критик саентологии, прошедший курс <<обучения>> под руководством <<известных специалистов>> в саентологии может превратиться в практика, перестав быть критиком.

  Если студент оказывается неспособным принять наш образ мышления, мы, конечно, ставим неудовлетворительную оценку на экзамене. Если он преодолевает этот этап, но потом решает, что наши аргументы туманны или неверны, мы изгоняем его из своего братства, называя тупицей и неудачником.

  Конечно, это не доказывает нашей неправоты в том, что мы уверены, будто имеем надежный метод поиска объективных истин, но мы должны признать, что за пределами нашего клуба по интересам большая часть того, чем мы занимаемся, выглядит полной бессмыслицей. У нас нет способа убедить самоуверенного скептика в том, что вещи, о которых мы говорим имеют значение, не говоря уже об их актуальном существовании.
\end{quotation}

Невозможность убеждения скептика является не просто следствием сложного устройства математических конструкций. Человеку вовсе необязательно разбираться в том, как устроена машина, чтобы понимать, что эта машина делает. Ни один информатик или программист не будет жаловаться на то, что ему непонятен компьютер, хотя сложность больших вычислительных машин, созданных десятками людей, оставляет далеко позади все римановы и неримановы гиперквадраты. Также, мало кто мог услышать такую жалобу во времена, предшествовашие двадцатому столетию. Если бы <<самоуверенный скептик>> сказал, что он не понимает, что такое числа или геометрические фигуры, его бы просто послали подальше, имея на это хорошую причину.

Мне кажется, что математика непонятна для непосвященного человека потому, что она на самом деле непонятна и для самих математиков, в противном случае, они могли бы объяснить остальным хотя бы ее основания. Но основания математики --- это как раз то место, где коренится наша проблема. Современная математика основана на теории множеств, имеющей дело с сущностями, способными бросить вызов здравому смыслу. Несмотря на это математические объекты, лежащие на такой непрочной основе, убеждают исследователей в том, что они занимаются изучением <<реальных вещей>>. Еще более удивительно, что такое убеждение разделяют и те математики, которые занимаются непосредственно теорией множеств. Этому может быть лишь одно разумное объяснение: формализм теории множеств действительно \textit{имеет отношение} к некоторой реальности, которая --- как и всякая реальность --- довольно постижима, в то время как ее текущая интерпретация, основанная на понятии актуальной бесконечности не просто выглядит бессмыслицей, но и скорее всего является неправильной. Если это утверждение справедливо, возможно, что математики пользуются теоретико-множественной интуицией, основанной именно на этих реальных, а не формализованных в современном виде объектах.

Взглянем на теоретико-множественные основания с точки зрения их согласованности. Работая с множествами, исследователь имеет интуитивное представление, может быть даже интуитивную уверенность, что эти основания непротиворечивы, несмотря на то, что их непротиворечивость никогда не была доказана. Если задуматься, это выглядит странным. Аксиоматическая теория множеств Цермело-Френкеля базируется на одиннадцати аксиомах, большая часть которых далека от элементарности или примитивности. Будучи объединенными в одну систему, эти аксиомы формируют еще менее примитивную сушность. Сложно представить, чтобы наша интуиция могла помогла нам осознать непротиворечивость такой конструкции, не имея корней в области некоторых простых, примитивных и непротиворечивых концепций и истин. Здесь мы приходим к вере в то, что подобные примитивные и интуитивно постижимые истины должны существовать. Доказать непротиворечивость теории множеств --- значит выделить эти истины и выразить в их терминах базовые аксиомы ZF. Однако, из теоремы Гёделя мы знаем, что доказать непротиворечивость теории множеств в терминах этой же самой теории невозможно. Это означает, что примитивные истины, о которых мы ранее говорили и к утверждению о чьем существовании мы пришли, должны быть очень необычными, невыразимыми в терминах множеств, хотя теоретико-множественные основания математики пытаются убедить нас в том, в этих терминах мы можем выразить любой объект строгого математического доказательства. Теория, построенная на этих истинах, тоже должна быть <<необычной>>. Используя популярное среди физиков высказывание Нильса Бора, она должна быть <<достаточно сумасбродной>>.

Подобная <<сумасбродная>> теория развивается в настоящей книге. Она приводит к полному принятию формализма теории множеств, но перефразирует его с соответствии с принципами конструктивизма, используя идею лишь потенциальной, но не актуальной бесконечности. Наша теория обладает следующими свойствами.

\begin{enumerate}
\item Математика рассматривается в качестве ветви научного знания. Математические объекты представляются так же, как и объекты других наук, они являются абстрагированными феноменами реальности, в которой мы живем. Как и объекты любой другой науки, математические объекты обладают своими особенностями, что приводит к существенным различиям количественного характера, однако в отношении природы, методов поиска и достоверности полученного знания существенных различий между математикой и другими науками не возникает.
  \item В соответствии с современной философией науки\footnote{Лакатос}
\end{enumerate}
